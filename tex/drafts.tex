\documentclass[a4,12pt]{article}

\usepackage[russian]{babel}
\usepackage[utf8x]{inputenc}

\author{Гера Глинских}
\title{Заметки к статье по генетическим алгоритмам}

\usepackage{amsmath,amsfonts,amsthm}
\usepackage{hyperref}

\theoremstyle{remark}
\newtheorem{defi}{Определение}
\newtheorem{stmt}{Утверждение}

\newcommand{\bdd}{\mathcal{D}}
\newcommand{\bdp}{\mathcal{P}}
\newcommand{\bdb}{\mathcal{B}}
\newcommand{\bdi}{\mathcal{I}}

\begin{document}

\maketitle{}

\section{\href{https://github.com/glinskikhg/combinatorics-summer/blob/f136271012f137d1f71936810deeffd4e36ccef6/docs/talk_zrinski-1.pdf}{Доклад Зриньски}}

\begin{defi}
\textbf{Блок-дизайном $\bdd$ с параметрами $t-(v,k,\lambda)$} назовем конечную инцидентную структуру $(\bdp,\bdb,\bdi)$, где $\bdp$ (точки) и $\bdb$ (блоки) не пересекаются, $\bdi \subset \bdp \times \bdb$ и
\begin{enumerate}
    \item $|\bdp| = v$ и $1 < k < v - 1$,
    \item каждый блок $\bdb$ инцидентен ровно с $k$ точками $\bdp$,
    \item каждые $t$ точек $\bdp$ инцидентны ровно $\lambda$ блокам $\bdb$.
\end{enumerate}
\end{defi}



\end{document}